\documentclass{article}

\usepackage{amsmath}

\begin{document}

Esercizio 2
\begin{itemize}
\item \textbf{Calcolo potenziale elettrico in un punto}:
    \\Il potenziale elettrico in un punto è dato dalla somma dei potenziali (di altri punti) che agiscono su quel punto.
    \begin{itemize}
        \item 1 punto: $k_e\frac{q}{r}$
        \item 2 punti (somma): $k_e\frac{q_1}{r_1}+k_e\frac{q_2}{r_2}$
    \end{itemize}

\item \textbf{Calcolo campo elettrico} in un punto o \textbf{forza nulla} in un punto:
    Il campo elettrico in un punto è dato dalla somma dei campi elettrici (di altri punti) in quel punto.
    \begin{itemize}
        \item 1 punto: $k_e\frac{Q}{|r_1-r_2|^2}$
        \item 2 punti: $k_e\frac{Q_1}{|r_1-r_2|^2}+k_e\frac{Q_2}{|r_2-r_3|^2}$
    \end{itemize}
  
    Se \textbf{la forza è nulla} allora i campi elettrici che agiscono in quel punto si annullano, quindi la loro somma è uguale a $0$
\item calcolo del campo magnetico in un punto
\item forza totale su una carica che si muove con una velocità
\item calcolo del campo elettrico generato 

\end{itemize}

\begin{equation*}
  1 + 2 = 3 +++
\end{equation*}

\begin{equation*}
  1 = 3 - 2
\end{equation*}

\begin{align*}
  1 + 2 &= 3\\
  1 &= 3 - 2
\end{align*}

\end{document}