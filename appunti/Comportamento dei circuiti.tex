\documentclass[17pt]{extarticle}

\usepackage{amsmath}
\usepackage{geometry}
% \geometry{margin=.7in}

\begin{document}

\section{Condensatori}
\begin{itemize}
    \item In condizioni stazionarie, si comportano come circuiti aperti.
    \item In condizioni non stazionarie, si comportano come circuiti chiusi.
    \item Se si è in condizioni stazionarie, i condensatori si caricano e hanno una certa differenza di potenziale.
    \item Nell’istante che il circuito non sia più stazionario, i condensatori si comportano come circuiti chiusi e iniziano a scaricarsi. Essi fungono come f.e.m. La ddp rimane la medesima. Il verso della corrente va dal polo positivo del condensatore.
    \item La ddp si calcola considerando la maglia relativa: si calcola la corrente presente nel ramo condiviso con un’altra maglia. Quando il circuito ritorna/diventa stazionario, i condensatori sono ormai scarichi, e la loro differenza di potenziale è pari a 0.
    \item Condensatori in serie: la carica presente sulle armature di $C_{eq}$ è la stessa carica presente sulle armature dei singoli condensatori.
    \item Per calcolare la ddp di un Condensatore in parallelo con una Resistenza, basta calcolare la ddp della Resistenza (i.e. $iR$)
    \item Per calcolare la corrente di una Resistenza in parallelo con un Condensatore (e la ddp è nota) basta calcolare: $iR = V$. Se la $ddp = 0$, la corrente di una Resistenza sarà anch’essa 0.

\end{itemize}


\section{Induttori}
\begin{itemize}
    \item In condizioni stazionarie, si comportano come circuiti chiusi.
    \item In condizioni non stazionarie, si comportano come circuiti aperti. 
    \item (Nell’istante che il circuito non sia più stazionario, gli induttori si comportano come circuiti aperti e compare una ddp.)
    \item Questa \textbf{ddp è pari alla corrente che attraversava precedentemente l’induttore} = nell’induttore continua a circolare corrente.
\end{itemize}

\section{Extra}
\begin{itemize}
    \item \textbf{Cortocircuito} = un circuito che ha resistenza quasi a zero o nulla. È, anche, un circuito chiuso perché la corrente circola.
    \item R $||$ cortocircuito = cortocircuito (resistore=0)
    \item R serie cortocircuito = prevale Resistore
    \item R $||$ circuito aperto = prevale Resistore
    \item R serie circuito aperto = circuito aperto
    \item R $||$ R nullo = prevale Resistore nullo
\end{itemize}

$ddp$ è in VOLT\\
Potenza è in AMPERE $(somma delle correnti)^2$ = potenza dissipata nel circuito
\begin{equation*}
    fem*i
\end{equation*}

\subsubsection*{SERIES}
\begin{itemize}
    \item STESSA $I$
    \item Diverso $V$
\end{itemize}

\subsubsection*{PARALLELO}
\begin{itemize}
    \item STESSO $V$
    \item DIVERSA $I$
\end{itemize}

$i$ va da + a -
\end{document}