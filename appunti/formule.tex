\documentclass[14pt]{extarticle}

\usepackage{amsmath}
\usepackage{geometry}
\geometry{margin=.7in}

\begin{document}


Cinematica\\
$a_c$ : accelerazione centripeta (\textbf{punta} costantemente verso il centro)\\
$w$ : velocità angolare\\
$r$ : raggio, eventualmente da calcolare. Cioè il modulo del vettore $|\overrightarrow{r}_{AB}|$  tra le particelle $A$ e $B$
\begin{equation*}
    \overrightarrow{w}=\frac{\left | \overrightarrow v \right |}{r}
    \quad\quad
    w=\sqrt{\frac{a_c}{r}}
\end{equation*}
\begin{equation*}
    v=wr
    \quad\quad
    v=\sqrt{a_cr}
    \quad\quad
    \overrightarrow{v}=\overrightarrow{w}\overrightarrow{r}
\end{equation*}
\begin{equation*}
    a_c=w^2r
    \quad\quad
    a_c=\frac{v^2}{r}
    \quad\quad
    \overrightarrow{a_c}=\frac{\overrightarrow{v^2}}{r}
\end{equation*}


Equazione moto rettilineo uniforme (quando $\overrightarrow{F}=0$)
\begin{equation*}
    s=v(t-t_i)+s_i
    \quad\quad
    z(t)=h-v_0t
\end{equation*}

Equazione moto circolare uniforme (quando $\overrightarrow{F}\perp\overrightarrow{v}$)
\begin{equation*}
    raggio = \frac{mv_o}{qB}
\end{equation*}

Componenti vettori
\begin{equation*}
    \alpha x= \left | \overrightarrow{a} \right |cos\alpha
\end{equation*}
\begin{equation*}
    \alpha y= \left | \overrightarrow{a} \right |sin\alpha
\end{equation*}

Velocità di una carica da A a B
\begin{equation*}
    \frac{m}{2}v^2=q(V_A-V_B)
\end{equation*}

\begin{equation*}
    i=\frac{\epsilon_i}{R}
    \quad\quad
    f=\frac{w}{2\pi}
    \quad\quad
    i=qf
    \quad\quad
    i^2=\frac{v^2}{R}
    \quad\quad
    w=\frac{v^2}{R_{tot}}
\end{equation*}

$\overrightarrow{B}$ per far muovere la carica
\begin{equation*}
    m\overrightarrow{a_c}=q\overrightarrow{v}\overrightarrow{B}
    \quad\quad
    \overrightarrow{B}=\frac{mw^2r}{qwr}
\end{equation*}

$\overrightarrow{B}$ generato da una carica in un punto
\begin{equation*}
    \overrightarrow{B}=\frac{km}{k_e}\overrightarrow{v}\overrightarrow{E}
\end{equation*}



% mie
Due cariche che interagiscono tra di loro, risentono di una forza $F$ che è proporzionale al valore delle cariche stesse ($q1$ e $q2$) e inversamente proporzionale al quadrato della distanza ($d$) tra le cariche:
\begin{equation*}
    \frac{q_1*q_2}{d^2}
\end{equation*}
\begin{center}
    \line(1,0){250}
\end{center}
Forza (elettrostatica) che $q1$ sente per la presenza di $q2$:
\begin{equation*}
    \overrightarrow{F}_{q_1,q_2}=k_e\frac{q_1q_2}{|d^2|}\frac{\overrightarrow{d}}{|d|}
\end{equation*}
\begin{itemize}
    \item attenzione al senso del vettore
    \item siccome $q1$ sente la forza, la direzione del vettore ($d$) sarà $q_2\rightarrow q_1$ : $\overrightarrow{r}_{q_2,q_1}$
\end{itemize}


\begin{center}
    \line(1,0){250}
\end{center}
Campo elettrico ($E$) prodotto in $q0$ da una carica puntiforme $q1$, distanti $d$
\begin{equation*}
    E_{q_o}=\frac{F_{q_o,q_1}}{q_0}=k_e\frac{\frac{q_oq_1}{d^2}}{q_0}=k_e\frac{q_1}{d^2}
\end{equation*}

Energia potenziale $U$ (lavoro $L$ da compiere da parte di una forza per portare una carica di prova $q$ ad una distanza $d$ dalla carica fissa $Q$)
\begin{equation*}
    L=U=qEd
\end{equation*}


Potenziale elettrico V di q
\begin{equation*}
    V=\frac{U}{q}
\end{equation*}

Potenziale totale in un punto $p$ generato da più cariche $q$ è la somma dei potenziali di tutte le cariche in quel punto + il potenziale all'infinito
\begin{align*}
    V_{tot} = k_e\frac{q_1}{d}+k_e\frac{q_2}{d}+V(\infty) \\
    \quad\text{dove $d$ è la distanza tra le cariche ed il punto $p$}
\end{align*}

Campo generato da più di una carica
\begin{equation*}
    \overrightarrow{F}=\overrightarrow{F_1}+\overrightarrow{F_2} \quad\text{con il vettore campo}
\end{equation*}
\begin{equation*}
    \overrightarrow{E}=\frac{\overrightarrow{E}}{q}=\frac{\overrightarrow{F_1}+\overrightarrow{F_2}}{q}=\frac{\overrightarrow{F_1}}{q}+\frac{\overrightarrow{F_2}}{q}=\overrightarrow{E_1}+\overrightarrow{E_2}
\end{equation*}

Lavoro fatto dal campo elettrico per portare una carica $Q$ da un punto A ad un punto B
\begin{equation*}
    L_{AB}=Q(V_A-V_B)
\end{equation*}

Angolo
\begin{equation*}
    tan^{-1}(\frac{y}{x})
\end{equation*}

\begin{center}
    \line(1,0){250}
\end{center}
Punto in cui la forza totale è nulla = Campo elettrico è nullo
\begin{center}
    \line(1,0){250}
\end{center}
\begin{itemize}
    \item Carica \textbf{ferma} in campo magnetico : \textbf{nessuna forza su di essa}
    \item Carica \textbf{in movimento} nella direzione del campo magnetico : \textbf{nessuna forza}, prod. cartesiano (0,0,0)
    \item Carica \textbf{in movimento} in direzione perpendicolare al campo magnetico : \textbf{interessata da forza}, la forza esercitata dal campo sulla carica sarà massima 
    quando la particella si muove in direzione perpendicolare al campo.
    \item Carica \textbf{in movimento} in campo magnetico: \textbf{forza} in relazione con il valore della carica, della sua velocità e anche della direzione in cui
    si muove secondo la legge:
\end{itemize}
\begin{equation*}
    \overrightarrow{F}=q\overrightarrow{v}\times\overrightarrow{B}=qvBsin\theta
\end{equation*}

\begin{center}
    \line(1,0){250}
\end{center}
Prodotto cartesiano
\begin{align*}
    a\times b=c\\
    c_x=a_yb_z-a_zb_y\\
    c_y=a_zb_x-a_xb_z\\
    c_z=a_xb_y-a_yb_z
\end{align*}
"Regola mano destra"
\begin{center}
    $a \rightarrow$ indice\\
    $b \rightarrow$ medio\\
    $a\times b \rightarrow$ pollice
\end{center}
\begin{center}
    \line(1,0){250}
\end{center}

Per calcolare il modulo del campo magnetico prodotto in un punto $p$ dal moto di una carica $q$
\begin{enumerate}
    \item calcolo $i$ in quel punto ($i=qf$ formule sopra)
    \item modulo del campo magnetico $B=2km\frac{i}{r}$ (formulario prof) 
\end{enumerate}
Se:
\begin{itemize}
    \item moto circolare: moto assimilabile a spira circolare percorsa da corrente
    \item moto rettilineo: moto assimilabile a spira dritta
\end{itemize}

    \begin{tabular}{l|l}
    C & coulomb \\ \hline
    m & metro   \\ \hline
    s & secondo \\ \hline
    j & joule   \\ \hline
    W & watt    \\ \hline
    N & newton  \\ \hline
    V & volt    \\ \hline
    A & ampere  \\ \hline
    F & farad   \\ \hline
    $\Omega$ & ohm    
    \end{tabular}





\begin{itemize}
    \item Forza elettrica: $\overrightarrow{F}=q\overrightarrow{E}$
    \item Potenziale elettrico di una carica: $V=\frac{k_eq}{r}$
    \item Energia Potenziale Elettrica: $U=\frac{k_eq_0q}{r}$
\end{itemize}

\begin{itemize}
    \item Energia immagazzinata in un Condensatore: $E=\frac{1}{2}QV=\frac{1}{2}CV^2=\frac{Q^2}{2C}$
    \item Capacità di un Condensatore: $C=\frac{Q}{V}$
    \item Campo magnetico: $B=\frac{I}{R}$
\end{itemize}

\begin{itemize}
    \item Forza di Coulomb (forza di un campo elettrico su una carica): $k_e\frac{q_1q_2}{r^2}$
    \item Campo Elettrico: $k_e\frac{Q}{|r_1-r_2|^2}$
    
\end{itemize}

\end{document}