\documentclass{article}

\usepackage{mathtools}

\begin{document}


% \begin{questions}
    % \question La legge di Coulomb ed il principio di sovrapposizione.  Come esempio si calcoli il campo di un dipolo elettrico sul  piano mediano del dipolo oppure lungo l'asse del dipolo.
    
% \end{questions}

\textbf{La legge di Coulomb ed il principio di sovrapposizione.  Come esempio si calcoli il campo di un dipolo elettrico sul  piano mediano del dipolo oppure lungo l'asse del dipolo.}
\\~\\
La legge di Coulomb afferma che tra due cariche elettriche $q_1$ e $q_2$ poste ad una distanza $d$ nel vuoto
si instaura una forza di natura elettrica, detta \textbf{forza di Coulomb}, il cui modulo è proporzionale al prodotto delle due cariche
e inversamente proporzionale al quadrato della distanza a cui esse sono poste:
\begin{equation}
    F = k_e \frac{q_1q_2}{d^2}
\end{equation}
Il principio di sovrapposizione dice che la forza risultante su di una carica è la \textbf{somma vettoriale}
di tutte le \textbf{forze elettriche} calcolate per ogni singola carica, come se le altre non ci fossero.
% \begin{equation}
%     \overrightarrow{F} = k_e \frac{q_1q_2}{d^2} \hat{r}
% \end{equation}

\end{document}